\documentclass[]{article}
\usepackage{lmodern}
\usepackage{amssymb,amsmath}
\usepackage{ifxetex,ifluatex}
\usepackage{fixltx2e} % provides \textsubscript
\ifnum 0\ifxetex 1\fi\ifluatex 1\fi=0 % if pdftex
  \usepackage[T1]{fontenc}
  \usepackage[utf8]{inputenc}
\else % if luatex or xelatex
  \ifxetex
    \usepackage{mathspec}
  \else
    \usepackage{fontspec}
  \fi
  \defaultfontfeatures{Ligatures=TeX,Scale=MatchLowercase}
\fi
% use upquote if available, for straight quotes in verbatim environments
\IfFileExists{upquote.sty}{\usepackage{upquote}}{}
% use microtype if available
\IfFileExists{microtype.sty}{%
\usepackage{microtype}
\UseMicrotypeSet[protrusion]{basicmath} % disable protrusion for tt fonts
}{}
\usepackage[margin=1in]{geometry}
\usepackage{hyperref}
\hypersetup{unicode=true,
            pdftitle={Homework 2:},
            pdfauthor={Sarafa Iyaniwura},
            pdfborder={0 0 0},
            breaklinks=true}
\urlstyle{same}  % don't use monospace font for urls
\usepackage{color}
\usepackage{fancyvrb}
\newcommand{\VerbBar}{|}
\newcommand{\VERB}{\Verb[commandchars=\\\{\}]}
\DefineVerbatimEnvironment{Highlighting}{Verbatim}{commandchars=\\\{\}}
% Add ',fontsize=\small' for more characters per line
\usepackage{framed}
\definecolor{shadecolor}{RGB}{248,248,248}
\newenvironment{Shaded}{\begin{snugshade}}{\end{snugshade}}
\newcommand{\KeywordTok}[1]{\textcolor[rgb]{0.13,0.29,0.53}{\textbf{{#1}}}}
\newcommand{\DataTypeTok}[1]{\textcolor[rgb]{0.13,0.29,0.53}{{#1}}}
\newcommand{\DecValTok}[1]{\textcolor[rgb]{0.00,0.00,0.81}{{#1}}}
\newcommand{\BaseNTok}[1]{\textcolor[rgb]{0.00,0.00,0.81}{{#1}}}
\newcommand{\FloatTok}[1]{\textcolor[rgb]{0.00,0.00,0.81}{{#1}}}
\newcommand{\ConstantTok}[1]{\textcolor[rgb]{0.00,0.00,0.00}{{#1}}}
\newcommand{\CharTok}[1]{\textcolor[rgb]{0.31,0.60,0.02}{{#1}}}
\newcommand{\SpecialCharTok}[1]{\textcolor[rgb]{0.00,0.00,0.00}{{#1}}}
\newcommand{\StringTok}[1]{\textcolor[rgb]{0.31,0.60,0.02}{{#1}}}
\newcommand{\VerbatimStringTok}[1]{\textcolor[rgb]{0.31,0.60,0.02}{{#1}}}
\newcommand{\SpecialStringTok}[1]{\textcolor[rgb]{0.31,0.60,0.02}{{#1}}}
\newcommand{\ImportTok}[1]{{#1}}
\newcommand{\CommentTok}[1]{\textcolor[rgb]{0.56,0.35,0.01}{\textit{{#1}}}}
\newcommand{\DocumentationTok}[1]{\textcolor[rgb]{0.56,0.35,0.01}{\textbf{\textit{{#1}}}}}
\newcommand{\AnnotationTok}[1]{\textcolor[rgb]{0.56,0.35,0.01}{\textbf{\textit{{#1}}}}}
\newcommand{\CommentVarTok}[1]{\textcolor[rgb]{0.56,0.35,0.01}{\textbf{\textit{{#1}}}}}
\newcommand{\OtherTok}[1]{\textcolor[rgb]{0.56,0.35,0.01}{{#1}}}
\newcommand{\FunctionTok}[1]{\textcolor[rgb]{0.00,0.00,0.00}{{#1}}}
\newcommand{\VariableTok}[1]{\textcolor[rgb]{0.00,0.00,0.00}{{#1}}}
\newcommand{\ControlFlowTok}[1]{\textcolor[rgb]{0.13,0.29,0.53}{\textbf{{#1}}}}
\newcommand{\OperatorTok}[1]{\textcolor[rgb]{0.81,0.36,0.00}{\textbf{{#1}}}}
\newcommand{\BuiltInTok}[1]{{#1}}
\newcommand{\ExtensionTok}[1]{{#1}}
\newcommand{\PreprocessorTok}[1]{\textcolor[rgb]{0.56,0.35,0.01}{\textit{{#1}}}}
\newcommand{\AttributeTok}[1]{\textcolor[rgb]{0.77,0.63,0.00}{{#1}}}
\newcommand{\RegionMarkerTok}[1]{{#1}}
\newcommand{\InformationTok}[1]{\textcolor[rgb]{0.56,0.35,0.01}{\textbf{\textit{{#1}}}}}
\newcommand{\WarningTok}[1]{\textcolor[rgb]{0.56,0.35,0.01}{\textbf{\textit{{#1}}}}}
\newcommand{\AlertTok}[1]{\textcolor[rgb]{0.94,0.16,0.16}{{#1}}}
\newcommand{\ErrorTok}[1]{\textcolor[rgb]{0.64,0.00,0.00}{\textbf{{#1}}}}
\newcommand{\NormalTok}[1]{{#1}}
\usepackage{longtable,booktabs}
\usepackage{graphicx,grffile}
\makeatletter
\def\maxwidth{\ifdim\Gin@nat@width>\linewidth\linewidth\else\Gin@nat@width\fi}
\def\maxheight{\ifdim\Gin@nat@height>\textheight\textheight\else\Gin@nat@height\fi}
\makeatother
% Scale images if necessary, so that they will not overflow the page
% margins by default, and it is still possible to overwrite the defaults
% using explicit options in \includegraphics[width, height, ...]{}
\setkeys{Gin}{width=\maxwidth,height=\maxheight,keepaspectratio}
\IfFileExists{parskip.sty}{%
\usepackage{parskip}
}{% else
\setlength{\parindent}{0pt}
\setlength{\parskip}{6pt plus 2pt minus 1pt}
}
\setlength{\emergencystretch}{3em}  % prevent overfull lines
\providecommand{\tightlist}{%
  \setlength{\itemsep}{0pt}\setlength{\parskip}{0pt}}
\setcounter{secnumdepth}{0}
% Redefines (sub)paragraphs to behave more like sections
\ifx\paragraph\undefined\else
\let\oldparagraph\paragraph
\renewcommand{\paragraph}[1]{\oldparagraph{#1}\mbox{}}
\fi
\ifx\subparagraph\undefined\else
\let\oldsubparagraph\subparagraph
\renewcommand{\subparagraph}[1]{\oldsubparagraph{#1}\mbox{}}
\fi

%%% Use protect on footnotes to avoid problems with footnotes in titles
\let\rmarkdownfootnote\footnote%
\def\footnote{\protect\rmarkdownfootnote}

%%% Change title format to be more compact
\usepackage{titling}

% Create subtitle command for use in maketitle
\newcommand{\subtitle}[1]{
  \posttitle{
    \begin{center}\large#1\end{center}
    }
}

\setlength{\droptitle}{-2em}

  \title{Homework 2:}
    \pretitle{\vspace{\droptitle}\centering\huge}
  \posttitle{\par}
    \author{Sarafa Iyaniwura}
    \preauthor{\centering\large\emph}
  \postauthor{\par}
      \predate{\centering\large\emph}
  \postdate{\par}
    \date{September 20, 2018}

\usepackage{booktabs}
\usepackage{longtable}
\usepackage{array}
\usepackage{multirow}
\usepackage[table]{xcolor}
\usepackage{wrapfig}
\usepackage{float}
\usepackage{colortbl}
\usepackage{pdflscape}
\usepackage{tabu}
\usepackage{threeparttable}
\usepackage{threeparttablex}
\usepackage[normalem]{ulem}
\usepackage{makecell}

\begin{document}
\maketitle

{
\setcounter{tocdepth}{2}
\tableofcontents
}
\section{\texorpdfstring{Exploring Gapminder data using \textbf{dplyr}
and
\textbf{ggplot2}}{Exploring Gapminder data using dplyr and ggplot2}}\label{exploring-gapminder-data-using-dplyr-and-ggplot2}

In this assignment, we explore the gapminder data set using the data
wranggling library \textbf{dplyr} and data exploration library
\textbf{ggplot2}, used for plotting data. The \textbf{dplyr} library is
loaded through the \textbf{tidyverse} library.

\subsection{1. Loading libraries}\label{loading-libraries}

First, let us load the libraries that would be used to explore the data
set. Here, we load the \textbf{gapminder} data set and the
\textbf{tidyverse} library.

\begin{Shaded}
\begin{Highlighting}[]
\KeywordTok{library}\NormalTok{(gapminder)  }\CommentTok{# loads gapminder data}
\KeywordTok{library}\NormalTok{(tidyverse)  }\CommentTok{# loads the tidyverse library}
\end{Highlighting}
\end{Shaded}

\begin{verbatim}
## Warning: replacing previous import by 'tibble::as_tibble' when loading
## 'broom'
\end{verbatim}

\begin{verbatim}
## Warning: replacing previous import by 'tibble::tibble' when loading 'broom'
\end{verbatim}

\begin{verbatim}
## -- Attaching packages ---------------------------------- tidyverse 1.2.1 --
\end{verbatim}

\begin{verbatim}
## √ ggplot2 3.0.0     √ purrr   0.2.5
## √ tibble  1.4.2     √ dplyr   0.7.6
## √ tidyr   0.8.1     √ stringr 1.3.1
## √ readr   1.1.1     √ forcats 0.3.0
\end{verbatim}

\begin{verbatim}
## -- Conflicts ------------------------------------- tidyverse_conflicts() --
## x dplyr::filter() masks stats::filter()
## x dplyr::lag()    masks stats::lag()
\end{verbatim}

\subsection{2. Smell testing the data: exploring the features of the
data}\label{smell-testing-the-data-exploring-the-features-of-the-data}

In this section, we shall check the features of the data such as the
structure, the number and data types of the variables, the dimension of
the data, among others.

\subsubsection{\texorpdfstring{2.1 \textbf{str()}
function}{2.1 str() function}}\label{str-function}

This function is used to check the structure and class of the data. It
give a comprehensive information about the data. Let us see how this
function is used and the output it produces.

\begin{Shaded}
\begin{Highlighting}[]
\KeywordTok{str}\NormalTok{(gapminder)  }\CommentTok{# displays the structure of the data like the variables, their types, the dimension, ...}
\end{Highlighting}
\end{Shaded}

\begin{verbatim}
## Classes 'tbl_df', 'tbl' and 'data.frame':    1704 obs. of  6 variables:
##  $ country  : Factor w/ 142 levels "Afghanistan",..: 1 1 1 1 1 1 1 1 1 1 ...
##  $ continent: Factor w/ 5 levels "Africa","Americas",..: 3 3 3 3 3 3 3 3 3 3 ...
##  $ year     : int  1952 1957 1962 1967 1972 1977 1982 1987 1992 1997 ...
##  $ lifeExp  : num  28.8 30.3 32 34 36.1 ...
##  $ pop      : int  8425333 9240934 10267083 11537966 13079460 14880372 12881816 13867957 16317921 22227415 ...
##  $ gdpPercap: num  779 821 853 836 740 ...
\end{verbatim}

Observe that this function has given us a lot of information about the
data. Information such as the class of the data, the number of
variables, the type of each of the variable and so on. It is important
to mention that one can still check for this properties of the data
using some other functions.

Before we use other functions to explore the structure of thiz data, let
us view its first few rows using the \textbf{head()} function

\begin{Shaded}
\begin{Highlighting}[]
\KeywordTok{head}\NormalTok{(gapminder) }\CommentTok{# displays the first few rows of the data (default is 6)}
\end{Highlighting}
\end{Shaded}

\begin{verbatim}
## # A tibble: 6 x 6
##   country     continent  year lifeExp      pop gdpPercap
##   <fct>       <fct>     <int>   <dbl>    <int>     <dbl>
## 1 Afghanistan Asia       1952    28.8  8425333      779.
## 2 Afghanistan Asia       1957    30.3  9240934      821.
## 3 Afghanistan Asia       1962    32.0 10267083      853.
## 4 Afghanistan Asia       1967    34.0 11537966      836.
## 5 Afghanistan Asia       1972    36.1 13079460      740.
## 6 Afghanistan Asia       1977    38.4 14880372      786.
\end{verbatim}

This function gave us the first 6 rows of the data, we can always get
more by using

\begin{Shaded}
\begin{Highlighting}[]
\KeywordTok{head}\NormalTok{(gapminder,}\DecValTok{15}\NormalTok{) }\CommentTok{# displays the first 15 rows of the gapminder data}
\end{Highlighting}
\end{Shaded}

\begin{verbatim}
## # A tibble: 15 x 6
##    country     continent  year lifeExp      pop gdpPercap
##    <fct>       <fct>     <int>   <dbl>    <int>     <dbl>
##  1 Afghanistan Asia       1952    28.8  8425333      779.
##  2 Afghanistan Asia       1957    30.3  9240934      821.
##  3 Afghanistan Asia       1962    32.0 10267083      853.
##  4 Afghanistan Asia       1967    34.0 11537966      836.
##  5 Afghanistan Asia       1972    36.1 13079460      740.
##  6 Afghanistan Asia       1977    38.4 14880372      786.
##  7 Afghanistan Asia       1982    39.9 12881816      978.
##  8 Afghanistan Asia       1987    40.8 13867957      852.
##  9 Afghanistan Asia       1992    41.7 16317921      649.
## 10 Afghanistan Asia       1997    41.8 22227415      635.
## 11 Afghanistan Asia       2002    42.1 25268405      727.
## 12 Afghanistan Asia       2007    43.8 31889923      975.
## 13 Albania     Europe     1952    55.2  1282697     1601.
## 14 Albania     Europe     1957    59.3  1476505     1942.
## 15 Albania     Europe     1962    64.8  1728137     2313.
\end{verbatim}

Also, the last few rows of a data frame can be displayed by using the
\textbf{tail()} function:

\begin{Shaded}
\begin{Highlighting}[]
\KeywordTok{tail}\NormalTok{(gapminder) }\CommentTok{# displays the last few rows of the data (default is 6)}
\end{Highlighting}
\end{Shaded}

\begin{verbatim}
## # A tibble: 6 x 6
##   country  continent  year lifeExp      pop gdpPercap
##   <fct>    <fct>     <int>   <dbl>    <int>     <dbl>
## 1 Zimbabwe Africa     1982    60.4  7636524      789.
## 2 Zimbabwe Africa     1987    62.4  9216418      706.
## 3 Zimbabwe Africa     1992    60.4 10704340      693.
## 4 Zimbabwe Africa     1997    46.8 11404948      792.
## 5 Zimbabwe Africa     2002    40.0 11926563      672.
## 6 Zimbabwe Africa     2007    43.5 12311143      470.
\end{verbatim}

Similarly, \textbf{tail(gapminder,n)} can be used to display the last
\(n\) rows of the gapminder data.

Now, let us check the features of the gapminder data one after the
other.

\subsubsection{\texorpdfstring{2.2 \textbf{class()}
function}{2.2 class() function}}\label{class-function}

\begin{Shaded}
\begin{Highlighting}[]
\KeywordTok{class}\NormalTok{(gapminder)  }\CommentTok{# displays the class of the data}
\end{Highlighting}
\end{Shaded}

\begin{verbatim}
## [1] "tbl_df"     "tbl"        "data.frame"
\end{verbatim}

\begin{enumerate}
\def\labelenumi{\arabic{enumi}.}
\tightlist
\item
  \emph{tbl\_df}: tibbles data frame
\item
  \emph{tbl} : tibbles
\item
  \emph{data.frame}: data frame
\end{enumerate}

This tells us that the gapminder data is a data frame.

\subsubsection{\texorpdfstring{2.3 \textbf{dim(), nrow(), ncol(), and
names()}
functions}{2.3 dim(), nrow(), ncol(), and names() functions}}\label{dim-nrow-ncol-and-names-functions}

The next few function are used to display the dimension, the number of
rows and columns, and the variables in the data.

\begin{Shaded}
\begin{Highlighting}[]
\KeywordTok{dim}\NormalTok{(gapminder) }\CommentTok{#  diplays  the dimension of data}
\end{Highlighting}
\end{Shaded}

\begin{verbatim}
## [1] 1704    6
\end{verbatim}

\begin{Shaded}
\begin{Highlighting}[]
\KeywordTok{nrow}\NormalTok{(gapminder) }\CommentTok{#  diplays  the number of rows}
\end{Highlighting}
\end{Shaded}

\begin{verbatim}
## [1] 1704
\end{verbatim}

\begin{Shaded}
\begin{Highlighting}[]
\KeywordTok{ncol}\NormalTok{(gapminder) }\CommentTok{#  diplays  the number of columns}
\end{Highlighting}
\end{Shaded}

\begin{verbatim}
## [1] 6
\end{verbatim}

\begin{Shaded}
\begin{Highlighting}[]
\KeywordTok{names}\NormalTok{(gapminder) }\CommentTok{# diplays the varibles/fields in the data }
\end{Highlighting}
\end{Shaded}

\begin{verbatim}
## [1] "country"   "continent" "year"      "lifeExp"   "pop"       "gdpPercap"
\end{verbatim}

\subsubsection{\texorpdfstring{2.4 \textbf{sapply()}
function}{2.4 sapply() function}}\label{sapply-function}

This function applies a specific function to a group of variables. Let
us illusttrate how it works by using it to get the class of each of the
variables in the gapminder data.

We use \textbf{sapply()} function to check the class of the variables in
our data as follows.

\begin{Shaded}
\begin{Highlighting}[]
\KeywordTok{sapply}\NormalTok{(gapminder,class) }\CommentTok{# applies the 'class()' function to each variable in the gapminder data}
\end{Highlighting}
\end{Shaded}

\begin{verbatim}
##   country continent      year   lifeExp       pop gdpPercap 
##  "factor"  "factor" "integer" "numeric" "integer" "numeric"
\end{verbatim}

\begin{longtable}[]{@{}ll@{}}
\toprule
Variables & data class\tabularnewline
\midrule
\endhead
country & factor/categorical\tabularnewline
continent & factor/categorical\tabularnewline
year & numeric\tabularnewline
lifeExp & integer\tabularnewline
pop gdpPercap & numeric\tabularnewline
\bottomrule
\end{longtable}

As another example, let us use the \textbf{sapply()} function to gapply
the function \textbf{typeof()} to the gapminder data.

\begin{Shaded}
\begin{Highlighting}[]
\KeywordTok{sapply}\NormalTok{(gapminder,typeof) }\CommentTok{# applies the 'typeof()' function to each variable in the gapminder data}
\end{Highlighting}
\end{Shaded}

\begin{verbatim}
##   country continent      year   lifeExp       pop gdpPercap 
## "integer" "integer" "integer"  "double" "integer"  "double"
\end{verbatim}

Note that what we got is not what we are actually expecting, for
example, `country' is categorical data but it gives integer. This is
because the \textbf{typeof()} function returns the types of R objects.

Let us try these same explorations by piping into the functions.

\begin{Shaded}
\begin{Highlighting}[]
\NormalTok{gapminder %>%}\StringTok{  }\CommentTok{# loads the gapminder data (and the result is piped into the next function)}
\StringTok{  }\KeywordTok{sapply}\NormalTok{(class) }\CommentTok{# applies the class function on the loaded data}
\end{Highlighting}
\end{Shaded}

\begin{verbatim}
##   country continent      year   lifeExp       pop gdpPercap 
##  "factor"  "factor" "integer" "numeric" "integer" "numeric"
\end{verbatim}

\begin{Shaded}
\begin{Highlighting}[]
\NormalTok{gapminder %>%}\StringTok{ }\CommentTok{# loads the gapminder data (and the result is piped into the next function)}
\StringTok{  }\KeywordTok{sapply}\NormalTok{(typeof) }\CommentTok{# applies the typeof() function on the loaded data}
\end{Highlighting}
\end{Shaded}

\begin{verbatim}
##   country continent      year   lifeExp       pop gdpPercap 
## "integer" "integer" "integer"  "double" "integer"  "double"
\end{verbatim}

\subsection{3. Exploring individual
variable}\label{exploring-individual-variable}

\subsection{3.1 Qualitative data}\label{qualitative-data}

Let us begin by considering the \textbf{continent} variable. We first
explore this variable using bar chat.

\subsubsection{\texorpdfstring{3.1.1 \textbf{Bar
chart}}{3.1.1 Bar chart}}\label{bar-chart}

\begin{Shaded}
\begin{Highlighting}[]
\NormalTok{barchart <-}\StringTok{ }\NormalTok{gapminder %>%}\StringTok{  }\CommentTok{# loads the gapminder data}
\StringTok{  }\KeywordTok{ggplot}\NormalTok{(}\KeywordTok{aes}\NormalTok{(}\DataTypeTok{x=}\NormalTok{continent,}\DataTypeTok{fill=}\NormalTok{continent)) +}\StringTok{ }\CommentTok{# calls the ggplot function and specify the axis and fill}
\StringTok{    }\KeywordTok{geom_bar}\NormalTok{()  }\CommentTok{# specify the type of  plot}

\NormalTok{barchart }\CommentTok{#sdisplays the bar chart}
\end{Highlighting}
\end{Shaded}

\includegraphics{Homework2_files/figure-latex/unnamed-chunk-12-1.pdf}

This bar chart shows the number of observations available for each
continent. For instance, we have over 600 observation from Africa,
around 400 for Asia, and around 300 for America.

This information can be shown more precisely by extracting the
`continent' column from the data and summarizing it as shown below;

The \textbf{table()} and \textbf{summary()} function

\begin{Shaded}
\begin{Highlighting}[]
\NormalTok{sumAry <-}\StringTok{ }\NormalTok{gapminder %>%}\StringTok{  }\CommentTok{# loads gapminder data}
\StringTok{  }\KeywordTok{select}\NormalTok{(continent) %>%}\StringTok{ }\CommentTok{# extract the column for continents}
\StringTok{    }\KeywordTok{summary}\NormalTok{()   }\CommentTok{# present a summary of the result}
\NormalTok{sumAry}
\end{Highlighting}
\end{Shaded}

\begin{verbatim}
##     continent  
##  Africa  :624  
##  Americas:300  
##  Asia    :396  
##  Europe  :360  
##  Oceania : 24
\end{verbatim}

We can also use the \textbf{table()} function to show the same result.

\begin{Shaded}
\begin{Highlighting}[]
\NormalTok{cont_table <-}\StringTok{ }\NormalTok{gapminder %>%}\StringTok{  }\CommentTok{# loads gapminder data}
\StringTok{  }\KeywordTok{select}\NormalTok{(continent) %>%}\StringTok{ }\CommentTok{# extract the column for continents}
\StringTok{    }\KeywordTok{table}\NormalTok{()  }\CommentTok{# present the result in  a table}

\NormalTok{cont_table  }\CommentTok{# display the table}
\end{Highlighting}
\end{Shaded}

\begin{verbatim}
## .
##   Africa Americas     Asia   Europe  Oceania 
##      624      300      396      360       24
\end{verbatim}

The \textbf{table()} and \textbf{summary()} functions as used here give
the precise number of observations available for each continent. Next,
we plot this information with a pie chart.

\subsubsection{\texorpdfstring{3.1.2 \textbf{Pie
chart}}{3.1.2 Pie chart}}\label{pie-chart}

Let us display the information about the amount of observations
available for each continent using a simple pie chart.

\begin{Shaded}
\begin{Highlighting}[]
\KeywordTok{pie}\NormalTok{(cont_table) }\CommentTok{# use the table constructed earlier to plot a pie chart}
\end{Highlighting}
\end{Shaded}

\includegraphics{Homework2_files/figure-latex/unnamed-chunk-15-1.pdf}

We can also plot a pie chart using \textbf{ggplot()} function

\begin{Shaded}
\begin{Highlighting}[]
\NormalTok{piechart <-}\StringTok{ }\NormalTok{barchart +}\StringTok{ }\KeywordTok{coord_polar}\NormalTok{() }\CommentTok{# plot a pie chart with the ggplot function}
\NormalTok{piechart}
\end{Highlighting}
\end{Shaded}

\includegraphics{Homework2_files/figure-latex/unnamed-chunk-16-1.pdf}

Observe that the code used to generate this plot depends on the code we
used to plot bar chart earlier. In addition, this plot is not good for
our data because it does not show the information for Oceania continent
properly. Let us try another pie chart:

\begin{Shaded}
\begin{Highlighting}[]
\CommentTok{# plotting another type of pie chart}
\NormalTok{piechart2 <-}\StringTok{ }\NormalTok{barchart +}\StringTok{ }
\StringTok{  }\KeywordTok{coord_polar}\NormalTok{(}\StringTok{"y"}\NormalTok{,}\DataTypeTok{start=}\DecValTok{0}\NormalTok{) +}\StringTok{ }
\StringTok{    }\KeywordTok{scale_fill_brewer}\NormalTok{(}\DataTypeTok{palette =} \StringTok{"Dark2"}\NormalTok{) +}\StringTok{ }
\StringTok{      }\KeywordTok{theme_minimal}\NormalTok{()  }
\NormalTok{piechart2}
\end{Highlighting}
\end{Shaded}

\includegraphics{Homework2_files/figure-latex/unnamed-chunk-17-1.pdf}

This plot is better in the sense that it shows the information for all
the continent well, but it may take some time to understand. Let us
explore other types of pie charts available in ggplot.

\begin{Shaded}
\begin{Highlighting}[]
\CommentTok{# plotting another type of pie chart}
\NormalTok{piechart3 <-}\StringTok{ }\KeywordTok{ggplot}\NormalTok{(gapminder,}\KeywordTok{aes}\NormalTok{(}\DataTypeTok{x=}\StringTok{""}\NormalTok{,}\DataTypeTok{fill=}\KeywordTok{factor}\NormalTok{(continent))) +}\StringTok{ }\KeywordTok{geom_bar}\NormalTok{(}\DataTypeTok{width=}\DecValTok{1}\NormalTok{) +}
\StringTok{  }\KeywordTok{coord_polar}\NormalTok{(}\StringTok{"y"}\NormalTok{,}\DataTypeTok{start=}\DecValTok{0}\NormalTok{)  +}
\StringTok{    }\KeywordTok{scale_fill_brewer}\NormalTok{(}\DataTypeTok{palette =} \StringTok{"Dark2"}\NormalTok{)  +}
\StringTok{      }\KeywordTok{theme_minimal}\NormalTok{()  }

\NormalTok{piechart3}
\end{Highlighting}
\end{Shaded}

\includegraphics{Homework2_files/figure-latex/unnamed-chunk-18-1.pdf}

\subsubsection{\texorpdfstring{3.1.3 \textbf{Bullseye
chart}}{3.1.3 Bullseye chart}}\label{bullseye-chart}

Let us plot a bullseye chart

\begin{Shaded}
\begin{Highlighting}[]
\CommentTok{# plotting a bullseye chart}
\KeywordTok{ggplot}\NormalTok{(gapminder,}\KeywordTok{aes}\NormalTok{(}\DataTypeTok{x=}\StringTok{""}\NormalTok{,}\DataTypeTok{fill=}\KeywordTok{factor}\NormalTok{(continent))) +}\StringTok{ }\KeywordTok{geom_bar}\NormalTok{(}\DataTypeTok{width=}\DecValTok{1}\NormalTok{) +}
\StringTok{  }\KeywordTok{coord_polar}\NormalTok{()  +}
\StringTok{    }\KeywordTok{scale_fill_brewer}\NormalTok{(}\DataTypeTok{palette =} \StringTok{"Dark2"}\NormalTok{)  +}
\StringTok{      }\KeywordTok{theme_minimal}\NormalTok{()  }
\end{Highlighting}
\end{Shaded}

\includegraphics{Homework2_files/figure-latex/unnamed-chunk-19-1.pdf}

\subsubsection{\texorpdfstring{3.1.3
\textbf{filter()}}{3.1.3 filter()}}\label{filter}

We can also extract the observations for each continent using the
\textbf{filter()} function.

\begin{Shaded}
\begin{Highlighting}[]
\NormalTok{gapminder %>%}
\StringTok{  }\KeywordTok{filter}\NormalTok{(continent ==}\StringTok{ 'Africa'}\NormalTok{) %>%}\StringTok{  }\CommentTok{# extract the data for African countries only}
\StringTok{    }\KeywordTok{dim}\NormalTok{() }\CommentTok{# displays the dimension of the extracted data}
\end{Highlighting}
\end{Shaded}

\begin{verbatim}
## [1] 624   6
\end{verbatim}

This shows that we have 624 observation for African countries. We can
always do the same for other continents or countries too. As one more
example, let us check how many observations are from Nigeria.

\begin{Shaded}
\begin{Highlighting}[]
\NormalTok{gapminder %>%}
\StringTok{  }\KeywordTok{filter}\NormalTok{(continent ==}\StringTok{ 'Africa'}\NormalTok{) %>%}\StringTok{  }\CommentTok{# extract the data for African countries only}
\StringTok{    }\KeywordTok{filter}\NormalTok{(country ==}\StringTok{ 'Nigeria'}\NormalTok{) %>%}\StringTok{ }\CommentTok{# extract the observations from Nigeria.}
\StringTok{      }\KeywordTok{dim}\NormalTok{() }\CommentTok{# displays the dimension of the extracted data}
\end{Highlighting}
\end{Shaded}

\begin{verbatim}
## [1] 12  6
\end{verbatim}

There are 12 of them. We can bypass the part of this code that first
extracted the data for Africa, and just extract that of Nigeria directly
from the gapminder data. This is shown below;

\begin{Shaded}
\begin{Highlighting}[]
\NormalTok{gapminder %>%}
\StringTok{    }\KeywordTok{filter}\NormalTok{(country ==}\StringTok{ 'Nigeria'}\NormalTok{) %>%}\StringTok{ }\CommentTok{# extract the observations from Nigeria.}
\StringTok{      }\KeywordTok{dim}\NormalTok{() }\CommentTok{# displays the dimension of the extracted data}
\end{Highlighting}
\end{Shaded}

\begin{verbatim}
## [1] 12  6
\end{verbatim}

\subsection{3.2 Quantitative data}\label{quantitative-data}

Here, we would be considering quantitative variables. First, let us
check how many years of data is available for each country.

\subsubsection{\texorpdfstring{3.2.1 \textbf{select()} and
\textbf{unique}
functions}{3.2.1 select() and unique functions}}\label{select-and-unique-functions}

\begin{Shaded}
\begin{Highlighting}[]
\NormalTok{gapminder %>%}
\StringTok{  }\KeywordTok{select}\NormalTok{(year) %>%}\StringTok{  }\CommentTok{# extracts the 'year' column}
\StringTok{    }\KeywordTok{unique}\NormalTok{() }\CommentTok{# displays each entry uniquely}
\end{Highlighting}
\end{Shaded}

\begin{verbatim}
## # A tibble: 12 x 1
##     year
##    <int>
##  1  1952
##  2  1957
##  3  1962
##  4  1967
##  5  1972
##  6  1977
##  7  1982
##  8  1987
##  9  1992
## 10  1997
## 11  2002
## 12  2007
\end{verbatim}

This shows that we have 12 years of data.

\subsubsection{\texorpdfstring{3.2.2 \textbf{summary()}
function}{3.2.2 summary() function}}\label{summary-function}

Let us look at the summary satistics of the numerical variables

\paragraph{\texorpdfstring{*
\textbf{Population}}{* Population}}\label{population}

\begin{Shaded}
\begin{Highlighting}[]
\NormalTok{gapminder %>%}
\StringTok{  }\KeywordTok{select}\NormalTok{(pop) %>%}\StringTok{ }\CommentTok{# extract population column}
\StringTok{    }\KeywordTok{summary}\NormalTok{()  }\CommentTok{# gives summary of population}
\end{Highlighting}
\end{Shaded}

\begin{verbatim}
##       pop           
##  Min.   :6.001e+04  
##  1st Qu.:2.794e+06  
##  Median :7.024e+06  
##  Mean   :2.960e+07  
##  3rd Qu.:1.959e+07  
##  Max.   :1.319e+09
\end{verbatim}

Here is the Summary statistics for population (pop)

\begin{longtable}[]{@{}ll@{}}
\toprule
Statistic & values\tabularnewline
\midrule
\endhead
Minimum & 60,010\tabularnewline
First quarter & 2.794 million\tabularnewline
Median & 7.024 million\tabularnewline
Mean & 29.60 million\tabularnewline
Third Quarter & 19.59 million\tabularnewline
Maximum & 1.319 billion\tabularnewline
\bottomrule
\end{longtable}

\paragraph{\texorpdfstring{* \textbf{Life
expectancy}}{* Life expectancy}}\label{life-expectancy}

\begin{Shaded}
\begin{Highlighting}[]
\NormalTok{gapminder %>%}
\StringTok{  }\KeywordTok{select}\NormalTok{(lifeExp) %>%}\StringTok{ }\CommentTok{# extract life expectancy column}
\StringTok{    }\KeywordTok{summary}\NormalTok{()  }\CommentTok{# displays the summary statistic for life expectancy}
\end{Highlighting}
\end{Shaded}

\begin{verbatim}
##     lifeExp     
##  Min.   :23.60  
##  1st Qu.:48.20  
##  Median :60.71  
##  Mean   :59.47  
##  3rd Qu.:70.85  
##  Max.   :82.60
\end{verbatim}

Here is the Summary statistics for Life expectancy

\begin{longtable}[]{@{}ll@{}}
\toprule
Statistic & values (years)\tabularnewline
\midrule
\endhead
Minimum & 23.60\tabularnewline
First quarter & 48.20\tabularnewline
Median & 60.71\tabularnewline
Mean & 59.47\tabularnewline
Third Quarter & 70.85\tabularnewline
Maximum & 82.60\tabularnewline
\bottomrule
\end{longtable}

\subsubsection{\texorpdfstring{3.2.3 \textbf{histogram()} and
\textbf{density()}}{3.2.3 histogram() and density()}}\label{histogram-and-density}

Let us plot the life expectancy for the extire gapminder data

\begin{Shaded}
\begin{Highlighting}[]
\NormalTok{gapminder %>%}\StringTok{ }\CommentTok{# loads the gapminder data}
\StringTok{  }\KeywordTok{ggplot}\NormalTok{(}\KeywordTok{aes}\NormalTok{(lifeExp)) +}\StringTok{ }\CommentTok{# calls the ggplot function}
\StringTok{    }\KeywordTok{geom_histogram}\NormalTok{(}\DataTypeTok{bins=}\DecValTok{30}\NormalTok{,}\KeywordTok{aes}\NormalTok{(}\DataTypeTok{fill=}\NormalTok{continent)) }\CommentTok{# specifies the type of plot}
\end{Highlighting}
\end{Shaded}

\includegraphics{Homework2_files/figure-latex/unnamed-chunk-26-1.pdf}

Hmm\ldots{} This plot does not give us detailed information about the
life expectancy for each continent. Let us plot the same data using a
density plot.

\begin{Shaded}
\begin{Highlighting}[]
\NormalTok{gapminder %>%}\StringTok{ }\CommentTok{# loads the gapminder data}
\StringTok{  }\KeywordTok{ggplot}\NormalTok{(}\KeywordTok{aes}\NormalTok{(lifeExp)) +}\StringTok{   }\CommentTok{# calls the ggplot function}
\StringTok{     }\KeywordTok{geom_density}\NormalTok{(}\KeywordTok{aes}\NormalTok{(}\DataTypeTok{fill=}\NormalTok{continent)) }\CommentTok{# specifies the type of plot}
\end{Highlighting}
\end{Shaded}

\includegraphics{Homework2_files/figure-latex/unnamed-chunk-27-1.pdf}

This looks better but we still have some overlapping of data. Maybe we
can plot for each continent separately?

\subsubsection{\texorpdfstring{3.2.4
\textbf{facet\_wrap()}}{3.2.4 facet\_wrap()}}\label{facet_wrap}

Now, let us plot the histogram and density plot for each continent
separately using the \textbf{facet\_wrap()} function.

\begin{Shaded}
\begin{Highlighting}[]
\NormalTok{gapminder %>%}\StringTok{  }\CommentTok{# loads the gapminder data}
\StringTok{  }\KeywordTok{ggplot}\NormalTok{(}\KeywordTok{aes}\NormalTok{(lifeExp)) +}\StringTok{ }\CommentTok{# calls the ggplot function}
\StringTok{    }\KeywordTok{geom_histogram}\NormalTok{(}\DataTypeTok{bins=}\DecValTok{30}\NormalTok{,}\KeywordTok{aes}\NormalTok{(}\DataTypeTok{fill=}\NormalTok{continent)) +}\StringTok{ }\CommentTok{# specifies the type of plot}
\StringTok{        }\KeywordTok{facet_wrap}\NormalTok{(~continent) }\CommentTok{# specifies that it should plot for each continent seperately}
\end{Highlighting}
\end{Shaded}

\includegraphics{Homework2_files/figure-latex/unnamed-chunk-28-1.pdf}

How about the gdp per capital for each continent?

\begin{Shaded}
\begin{Highlighting}[]
\NormalTok{gapminder %>%}\StringTok{  }\CommentTok{# loads the gapminder data}
\StringTok{  }\KeywordTok{ggplot}\NormalTok{(}\KeywordTok{aes}\NormalTok{(gdpPercap)) +}\StringTok{ }\CommentTok{# calls the ggplot function}
\StringTok{    }\KeywordTok{geom_histogram}\NormalTok{(}\DataTypeTok{bins=}\DecValTok{30}\NormalTok{,}\KeywordTok{aes}\NormalTok{(}\DataTypeTok{fill=}\NormalTok{continent)) +}\StringTok{  }\CommentTok{# specifies the type of plot}
\StringTok{      }\KeywordTok{facet_wrap}\NormalTok{(~continent) +}\CommentTok{# specifies that it should plot for each continent seperately}
\StringTok{        }\KeywordTok{scale_y_log10}\NormalTok{()}
\end{Highlighting}
\end{Shaded}

\begin{verbatim}
## Warning: Transformation introduced infinite values in continuous y-axis
\end{verbatim}

\begin{verbatim}
## Warning: Removed 94 rows containing missing values (geom_bar).
\end{verbatim}

\includegraphics{Homework2_files/figure-latex/unnamed-chunk-29-1.pdf}

\begin{Shaded}
\begin{Highlighting}[]
\NormalTok{gapminder %>%}\StringTok{  }\CommentTok{# loads the gapminder data}
\StringTok{  }\KeywordTok{ggplot}\NormalTok{(}\KeywordTok{aes}\NormalTok{(lifeExp)) +}\StringTok{ }\CommentTok{# calls the ggplot function}
\StringTok{    }\KeywordTok{geom_density}\NormalTok{(}\DataTypeTok{bins=}\DecValTok{30}\NormalTok{,}\KeywordTok{aes}\NormalTok{(}\DataTypeTok{fill=}\NormalTok{continent)) +}\StringTok{ }\CommentTok{# specifies the type of plot}
\StringTok{        }\KeywordTok{facet_wrap}\NormalTok{(~continent) }\CommentTok{# specifies that it should plot for each continent seperately}
\end{Highlighting}
\end{Shaded}

\begin{verbatim}
## Warning: Ignoring unknown parameters: bins
\end{verbatim}

\includegraphics{Homework2_files/figure-latex/unnamed-chunk-30-1.pdf}

\begin{Shaded}
\begin{Highlighting}[]
\NormalTok{gapminder %>%}\StringTok{  }\CommentTok{# loads the gapminder data}
\StringTok{  }\KeywordTok{ggplot}\NormalTok{(}\KeywordTok{aes}\NormalTok{(gdpPercap)) +}\StringTok{ }\CommentTok{# calls the ggplot function}
\StringTok{    }\KeywordTok{geom_density}\NormalTok{(}\DataTypeTok{bins=}\DecValTok{30}\NormalTok{,}\KeywordTok{aes}\NormalTok{(}\DataTypeTok{fill=}\NormalTok{continent)) +}\StringTok{ }\CommentTok{# specifies the type of plot}
\StringTok{        }\KeywordTok{facet_wrap}\NormalTok{(~continent) }\CommentTok{# specifies that it should plot for each continent seperately}
\end{Highlighting}
\end{Shaded}

\begin{verbatim}
## Warning: Ignoring unknown parameters: bins
\end{verbatim}

\includegraphics{Homework2_files/figure-latex/unnamed-chunk-31-1.pdf}

\subsection{4. Exploring various plots}\label{exploring-various-plots}

In this section, we shall be exploring and visualizing the entire
gapminder data or part of using plots and figures.

\subsubsection{4.1 Gdp per capital and
population}\label{gdp-per-capital-and-population}

Let us begin by plotting a scatter plot of the *gdpPercap\textbf{ and
}population**.

\begin{Shaded}
\begin{Highlighting}[]
\NormalTok{gapminder %>%}\StringTok{ }\CommentTok{# loads the gapminder data}
\StringTok{  }\KeywordTok{select}\NormalTok{(pop,gdpPercap,continent) %>%}\StringTok{ }\CommentTok{# extract the  columns to be considered}
\StringTok{    }\KeywordTok{ggplot}\NormalTok{(}\KeywordTok{aes}\NormalTok{(pop,gdpPercap)) +}\StringTok{  }\CommentTok{# calling the ggplot function}
\StringTok{        }\KeywordTok{geom_point}\NormalTok{(}\KeywordTok{aes}\NormalTok{(}\DataTypeTok{color=}\NormalTok{continent))  +}\StringTok{ }\CommentTok{# specifies the type of plot}
\StringTok{          }\KeywordTok{scale_y_log10}\NormalTok{() +}\StringTok{ }\KeywordTok{scale_x_log10}\NormalTok{() }\CommentTok{# log scale for both axis}
\end{Highlighting}
\end{Shaded}

\includegraphics{Homework2_files/figure-latex/unnamed-chunk-32-1.pdf}

Hmm this looks messy! Let us plot the same columns but for one continent
only, say Europe.

\begin{Shaded}
\begin{Highlighting}[]
\NormalTok{gapminder %>%}\StringTok{  }\CommentTok{# loads the gapminder data}
\StringTok{  }\KeywordTok{filter}\NormalTok{(continent==}\StringTok{'Europe'}\NormalTok{)  %>%}\StringTok{   }\CommentTok{# extracting data for Europian countries only}
\StringTok{    }\KeywordTok{select}\NormalTok{(pop,gdpPercap,continent) %>%}\StringTok{ }\CommentTok{# select the columns to be considered}
\StringTok{      }\KeywordTok{ggplot}\NormalTok{(}\KeywordTok{aes}\NormalTok{(pop,gdpPercap)) +}\StringTok{   }\CommentTok{# calling the ggplot function}
\StringTok{        }\KeywordTok{geom_point}\NormalTok{(}\KeywordTok{aes}\NormalTok{(}\DataTypeTok{color=}\NormalTok{continent))  +}\StringTok{ }\CommentTok{# specifies the type of plot}
\StringTok{          }\KeywordTok{scale_y_log10}\NormalTok{() +}\StringTok{ }\KeywordTok{scale_x_log10}\NormalTok{() }\CommentTok{# log scale for both axis}
\end{Highlighting}
\end{Shaded}

\includegraphics{Homework2_files/figure-latex/unnamed-chunk-33-1.pdf}

We can do this in a more fancy way by plotting for each continent
separately using the \textbf{facet\_wrap()} function.

\begin{Shaded}
\begin{Highlighting}[]
\NormalTok{gapminder %>%}\StringTok{ }\CommentTok{# loads the gapminder data}
\StringTok{      }\KeywordTok{select}\NormalTok{(pop,gdpPercap,continent)  %>%}\StringTok{  }\CommentTok{# select the variables to be considered}
\StringTok{        }\KeywordTok{ggplot}\NormalTok{(}\KeywordTok{aes}\NormalTok{(pop,gdpPercap)) +}\StringTok{   }\CommentTok{# calling the ggplot function}
\StringTok{          }\KeywordTok{geom_point}\NormalTok{(}\KeywordTok{aes}\NormalTok{(}\DataTypeTok{color=}\NormalTok{continent)) +}\StringTok{ }\CommentTok{# specifies the type of plot}
\StringTok{              }\KeywordTok{facet_wrap}\NormalTok{(~continent) +}\StringTok{ }
\StringTok{                  }\KeywordTok{scale_y_log10}\NormalTok{() +}\StringTok{ }\KeywordTok{scale_x_log10}\NormalTok{() }\CommentTok{# log scale for both axis}
\end{Highlighting}
\end{Shaded}

\includegraphics{Homework2_files/figure-latex/unnamed-chunk-34-1.pdf}

This shows the GDP per capital vs population for each continent.

Some of the things we can easily see from these plots is that in Asia,
when the pupolation is high, the gdp per capital is low and when the gdp
per capital is high, the population is low. From this graphs we can
easily see the relationship between the gdp per capital for each country
and the population over the years.

\subsubsection{4.1 Population and life
expectancy}\label{population-and-life-expectancy}

Let us plot population vs life expectancy for this data and colour them
by continent.

\begin{Shaded}
\begin{Highlighting}[]
\NormalTok{gapminder %>%}\StringTok{  }\CommentTok{# loads the gapminder data}
\StringTok{    }\KeywordTok{select}\NormalTok{(lifeExp,pop,continent) %>%}\StringTok{ }\CommentTok{# select the variables to be considered}
\StringTok{        }\KeywordTok{ggplot}\NormalTok{(}\KeywordTok{aes}\NormalTok{(}\DataTypeTok{x=}\NormalTok{lifeExp,}\DataTypeTok{y=}\NormalTok{pop)) +}\StringTok{   }\CommentTok{# calling the ggplot function}
\StringTok{            }\KeywordTok{geom_point}\NormalTok{(}\KeywordTok{aes}\NormalTok{(}\DataTypeTok{color=}\NormalTok{continent))+}\StringTok{  }\CommentTok{# specifies the type of plot}
\StringTok{                }\KeywordTok{scale_y_log10}\NormalTok{()}
\end{Highlighting}
\end{Shaded}

\includegraphics{Homework2_files/figure-latex/unnamed-chunk-35-1.pdf}

Now, let us plot this information continent by continent.

\begin{Shaded}
\begin{Highlighting}[]
\NormalTok{gapminder %>%}\StringTok{  }\CommentTok{# loads the gapminder data}
\StringTok{        }\KeywordTok{ggplot}\NormalTok{(}\KeywordTok{aes}\NormalTok{(lifeExp,pop)) +}\StringTok{   }\CommentTok{# calling the ggplot function}
\StringTok{          }\KeywordTok{geom_point}\NormalTok{(}\KeywordTok{aes}\NormalTok{(}\DataTypeTok{color=}\NormalTok{continent)) +}\StringTok{ }\CommentTok{# specifies the type of plot}
\StringTok{              }\KeywordTok{facet_wrap}\NormalTok{(~continent) +}\StringTok{ }\KeywordTok{scale_y_log10}\NormalTok{()}
\end{Highlighting}
\end{Shaded}

\includegraphics{Homework2_files/figure-latex/unnamed-chunk-36-1.pdf}

How about somthing similar for African and Europian countries only.

\begin{Shaded}
\begin{Highlighting}[]
\NormalTok{gapminder %>%}\StringTok{ }\CommentTok{# loads the gapminder data}
\StringTok{    }\KeywordTok{filter}\NormalTok{(continent==}\StringTok{'Europe'} \NormalTok{|}\StringTok{ }\NormalTok{continent==}\StringTok{'Africa'}\NormalTok{)  %>%}\StringTok{  }\CommentTok{# extract data for African and Europian countries only}
\StringTok{        }\KeywordTok{ggplot}\NormalTok{(}\KeywordTok{aes}\NormalTok{(lifeExp,pop)) +}\StringTok{  }\CommentTok{# calling the ggplot function}
\StringTok{          }\KeywordTok{geom_point}\NormalTok{(}\KeywordTok{aes}\NormalTok{(}\DataTypeTok{color=}\NormalTok{continent)) +}\StringTok{ }\CommentTok{# specifies the type of plot}
\StringTok{            }\KeywordTok{scale_y_log10}\NormalTok{()  }\CommentTok{# specify log scale for y}
\end{Highlighting}
\end{Shaded}

\includegraphics{Homework2_files/figure-latex/unnamed-chunk-37-1.pdf}

This shows that most of the African countries have lower life
expentancy, relative to the Europian countries.

Let us see how this information is displayed with a \textbf{boxplot()}
and a \textbf{violin()} plot:

\begin{itemize}
\tightlist
\item
  Boxplot
\end{itemize}

\begin{Shaded}
\begin{Highlighting}[]
\NormalTok{gapminder %>%}\StringTok{ }\CommentTok{# loads the gapminder data}
\StringTok{    }\KeywordTok{filter}\NormalTok{(continent==}\StringTok{'Europe'} \NormalTok{|}\StringTok{ }\NormalTok{continent==}\StringTok{'Africa'}\NormalTok{)  %>%}\StringTok{  }\CommentTok{# extract data for African and Europian countries only}
\StringTok{        }\KeywordTok{ggplot}\NormalTok{(}\KeywordTok{aes}\NormalTok{(lifeExp,pop,}\DataTypeTok{fill=}\NormalTok{continent)) +}\StringTok{  }\CommentTok{# calling the ggplot function}
\StringTok{          }\KeywordTok{geom_boxplot}\NormalTok{(}\KeywordTok{aes}\NormalTok{(}\DataTypeTok{color=}\NormalTok{continent)) +}\StringTok{ }\CommentTok{# specifies the type of plot}
\StringTok{            }\KeywordTok{scale_y_log10}\NormalTok{()  }\CommentTok{# specify log scale for y}
\end{Highlighting}
\end{Shaded}

\includegraphics{Homework2_files/figure-latex/unnamed-chunk-38-1.pdf}

\begin{itemize}
\tightlist
\item
  Violin plot
\end{itemize}

\begin{Shaded}
\begin{Highlighting}[]
\NormalTok{gapminder %>%}\StringTok{  }\CommentTok{# loads the gapminder data}
\StringTok{    }\KeywordTok{filter}\NormalTok{(continent==}\StringTok{'Europe'} \NormalTok{|}\StringTok{ }\NormalTok{continent==}\StringTok{'Africa'}\NormalTok{)  %>%}\StringTok{  }\CommentTok{# extract data for African and Europian countries only}
\StringTok{      }\KeywordTok{select}\NormalTok{(lifeExp,pop,continent)  %>%}
\StringTok{        }\KeywordTok{ggplot}\NormalTok{(}\KeywordTok{aes}\NormalTok{(lifeExp,pop,}\DataTypeTok{fill=}\NormalTok{continent)) +}
\StringTok{          }\KeywordTok{geom_violin}\NormalTok{(}\KeywordTok{aes}\NormalTok{(}\DataTypeTok{color=}\NormalTok{continent))  +}\StringTok{ }\CommentTok{# specifies the type of plot}
\StringTok{            }\KeywordTok{scale_y_log10}\NormalTok{()  }\CommentTok{# specify log scale for y}
\end{Highlighting}
\end{Shaded}

\begin{verbatim}
## Warning: position_dodge requires non-overlapping x intervals
\end{verbatim}

\includegraphics{Homework2_files/figure-latex/unnamed-chunk-39-1.pdf}

Now, let us use \textbf{boxplot()} to show the life expentancy data for
each continent.

\begin{Shaded}
\begin{Highlighting}[]
\NormalTok{gapminder %>%}\StringTok{ }\CommentTok{# loads the gapminder data}
\StringTok{  }\KeywordTok{ggplot}\NormalTok{(}\KeywordTok{aes}\NormalTok{(continent,lifeExp,}\DataTypeTok{fill=}\NormalTok{continent)) +}\StringTok{ }
\StringTok{    }\KeywordTok{geom_boxplot}\NormalTok{(}\KeywordTok{aes}\NormalTok{(}\DataTypeTok{color=}\NormalTok{continent))  }\CommentTok{# specifies the type of plot}
\end{Highlighting}
\end{Shaded}

\includegraphics{Homework2_files/figure-latex/unnamed-chunk-40-1.pdf}

What of using \textbf{geom\_violin()} plot together with
\textbf{geom\_jitter()} for the same data?

\begin{Shaded}
\begin{Highlighting}[]
\NormalTok{gapminder %>%}\StringTok{  }\CommentTok{# loads the gapminder data}
\StringTok{  }\KeywordTok{ggplot}\NormalTok{(}\KeywordTok{aes}\NormalTok{(continent,lifeExp)) +}\StringTok{ }
\StringTok{    }\KeywordTok{geom_violin}\NormalTok{(}\KeywordTok{aes}\NormalTok{(}\DataTypeTok{color=}\NormalTok{continent,}\DataTypeTok{fill=}\NormalTok{continent)) +}\StringTok{  }\CommentTok{# specifies the type of plot}
\StringTok{      }\KeywordTok{geom_jitter}\NormalTok{(}\DataTypeTok{alpha=}\FloatTok{0.2}\NormalTok{)  }\CommentTok{# specifies additional type of plot}
\end{Highlighting}
\end{Shaded}

\includegraphics{Homework2_files/figure-latex/unnamed-chunk-41-1.pdf}

\subsubsection{4.3 Gdp per capital and
continent}\label{gdp-per-capital-and-continent}

Let us use a boxplot to show the gpd per capital for each continent in
different years using \textbf{facet\_wrap()} function

\begin{Shaded}
\begin{Highlighting}[]
\NormalTok{gapminder %>%}\StringTok{  }\CommentTok{# loads the gapminder data}
\StringTok{  }\KeywordTok{ggplot}\NormalTok{(}\KeywordTok{aes}\NormalTok{(continent,gdpPercap)) +}
\StringTok{            }\KeywordTok{geom_boxplot}\NormalTok{(}\DataTypeTok{fill=}\StringTok{'green'}\NormalTok{) +}\StringTok{  }\CommentTok{# specifies the type of plot}
\StringTok{                }\KeywordTok{scale_y_log10}\NormalTok{() +}\StringTok{ }
\StringTok{                  }\KeywordTok{geom_jitter}\NormalTok{(}\DataTypeTok{alpha=}\FloatTok{0.3}\NormalTok{,}\DataTypeTok{fill=}\StringTok{'red'}\NormalTok{) +}\StringTok{   }\CommentTok{# specifies additional type of plot}
\StringTok{                        }\KeywordTok{facet_wrap}\NormalTok{(~year)  }\CommentTok{# specify that each  year should be plotted separately}
\end{Highlighting}
\end{Shaded}

\includegraphics{Homework2_files/figure-latex/unnamed-chunk-42-1.pdf}

How about plotting the same figure but for some selected years? Let us
do this for the following years; 1952, 1962, 1972, 1982, 1997, and 2007.

\begin{Shaded}
\begin{Highlighting}[]
\NormalTok{gapminder %>%}\StringTok{  }\CommentTok{# loads the gapminder data}
\StringTok{  }\KeywordTok{filter}\NormalTok{(year ==}\StringTok{ '1952'} \NormalTok{|}\StringTok{ }\NormalTok{year ==}\StringTok{ '1962'} \NormalTok{|}\StringTok{ }\NormalTok{year ==}\StringTok{ '1972'} \NormalTok{|}\StringTok{ }\NormalTok{year ==}\StringTok{ '1982'} \NormalTok{|}\StringTok{ }\NormalTok{year ==}\StringTok{ '1997'} \NormalTok{|}\StringTok{ }\NormalTok{year ==}\StringTok{ '2007'}\NormalTok{) %>%}
\StringTok{    }\KeywordTok{ggplot}\NormalTok{(}\KeywordTok{aes}\NormalTok{(continent,gdpPercap)) +}
\StringTok{              }\KeywordTok{geom_boxplot}\NormalTok{(}\DataTypeTok{fill=}\StringTok{'green'}\NormalTok{) +}
\StringTok{                  }\KeywordTok{scale_y_log10}\NormalTok{() +}\StringTok{ }
\StringTok{                    }\KeywordTok{geom_jitter}\NormalTok{(}\DataTypeTok{alpha=}\FloatTok{0.3}\NormalTok{,}\DataTypeTok{fill=}\StringTok{'red'}\NormalTok{) +}\StringTok{ }
\StringTok{    }\KeywordTok{facet_wrap}\NormalTok{(~year)}
\end{Highlighting}
\end{Shaded}

\includegraphics{Homework2_files/figure-latex/unnamed-chunk-43-1.pdf}

What if we want to do the same but for a specific year upward or
downward? Say from 1982 to 2007

\begin{Shaded}
\begin{Highlighting}[]
\NormalTok{gapminder %>%}\StringTok{ }\CommentTok{# loads the gapminder data}
\StringTok{  }\KeywordTok{filter}\NormalTok{(year >=}\StringTok{ "1982"}\NormalTok{) %>%}
\StringTok{    }\KeywordTok{ggplot}\NormalTok{(}\KeywordTok{aes}\NormalTok{(continent,gdpPercap)) +}
\StringTok{              }\KeywordTok{geom_boxplot}\NormalTok{(}\DataTypeTok{fill=}\StringTok{'green'}\NormalTok{) +}
\StringTok{                  }\KeywordTok{scale_y_log10}\NormalTok{() +}\StringTok{ }
\StringTok{                    }\KeywordTok{geom_jitter}\NormalTok{(}\DataTypeTok{alpha=}\FloatTok{0.3}\NormalTok{,}\DataTypeTok{fill=}\StringTok{'red'}\NormalTok{) +}\StringTok{ }
\StringTok{    }\KeywordTok{facet_wrap}\NormalTok{(~year)}
\end{Highlighting}
\end{Shaded}

\includegraphics{Homework2_files/figure-latex/unnamed-chunk-44-1.pdf}

Let us do the same for countreis less than 1982;

\begin{Shaded}
\begin{Highlighting}[]
\NormalTok{gapminder %>%}\StringTok{ }\CommentTok{# loads the gapminder data}
\StringTok{  }\KeywordTok{filter}\NormalTok{(year <}\StringTok{ '1982'}\NormalTok{) %>%}
\StringTok{    }\KeywordTok{ggplot}\NormalTok{(}\KeywordTok{aes}\NormalTok{(continent,gdpPercap)) +}
\StringTok{              }\KeywordTok{geom_boxplot}\NormalTok{(}\DataTypeTok{fill=}\StringTok{'green'}\NormalTok{) +}
\StringTok{                  }\KeywordTok{scale_y_log10}\NormalTok{() +}\StringTok{ }
\StringTok{                    }\KeywordTok{geom_jitter}\NormalTok{(}\DataTypeTok{alpha=}\FloatTok{0.3}\NormalTok{,}\DataTypeTok{fill=}\StringTok{'red'}\NormalTok{) +}\StringTok{ }
\StringTok{    }\KeywordTok{facet_wrap}\NormalTok{(~year)}
\end{Highlighting}
\end{Shaded}

\includegraphics{Homework2_files/figure-latex/unnamed-chunk-45-1.pdf}

Let us use \textbf{geom\_smooth()} function to plot the population for
all the continent.

\begin{Shaded}
\begin{Highlighting}[]
\NormalTok{gapminder %>%}\StringTok{ }\CommentTok{# loads the gapminder data}
\StringTok{    }\KeywordTok{ggplot}\NormalTok{( }\KeywordTok{aes}\NormalTok{(year,pop,}\DataTypeTok{colour=}\NormalTok{continent)) +}\StringTok{ }
\StringTok{      }\KeywordTok{geom_smooth}\NormalTok{() }
\end{Highlighting}
\end{Shaded}

\begin{verbatim}
## `geom_smooth()` using method = 'loess' and formula 'y ~ x'
\end{verbatim}

\includegraphics{Homework2_files/figure-latex/unnamed-chunk-46-1.pdf}

Population is increasing over the years.

How about the same for only Asia and Africa over the years?

\begin{Shaded}
\begin{Highlighting}[]
\NormalTok{gapminder %>%}\StringTok{ }\CommentTok{# loads the gapminder data}
\StringTok{  }\KeywordTok{filter}\NormalTok{(continent==}\StringTok{"Asia"} \NormalTok{|}\StringTok{ }\NormalTok{continent==}\StringTok{"Africa"}\NormalTok{) %>%}
\StringTok{    }\KeywordTok{ggplot}\NormalTok{( }\KeywordTok{aes}\NormalTok{(year,pop,}\DataTypeTok{colour=}\NormalTok{continent)) +}\StringTok{ }
\StringTok{      }\KeywordTok{geom_smooth}\NormalTok{() }
\end{Highlighting}
\end{Shaded}

\begin{verbatim}
## `geom_smooth()` using method = 'loess' and formula 'y ~ x'
\end{verbatim}

\includegraphics{Homework2_files/figure-latex/unnamed-chunk-47-1.pdf}

Asian's population is increasing faster than that of Africa over the
years.

Let us plot the gdp per capital for all the continents using the same
function.

\begin{Shaded}
\begin{Highlighting}[]
\NormalTok{gapminder %>%}\StringTok{ }\CommentTok{# loads the gapminder data}
\StringTok{  }\KeywordTok{ggplot}\NormalTok{( }\KeywordTok{aes}\NormalTok{(year,gdpPercap,}\DataTypeTok{colour=}\NormalTok{continent)) +}\StringTok{ }
\StringTok{    }\KeywordTok{geom_smooth}\NormalTok{(}\DataTypeTok{model=}\NormalTok{lm) }
\end{Highlighting}
\end{Shaded}

\begin{verbatim}
## Warning: Ignoring unknown parameters: model
\end{verbatim}

\begin{verbatim}
## `geom_smooth()` using method = 'loess' and formula 'y ~ x'
\end{verbatim}

\includegraphics{Homework2_files/figure-latex/unnamed-chunk-48-1.pdf}

Gpd per capital for each continent is increasing over the years, with
Africa having the slowest growth.

\subsection{4.1 Extracting part of the data and exploring
it.}\label{extracting-part-of-the-data-and-exploring-it.}

Here, I will be extracting the observations from African and then those
from Nigeria. We will explore the extracted data in more detail.

Let us extract the observations for all the African countries using the
\textbf{filter()} function

\begin{Shaded}
\begin{Highlighting}[]
\NormalTok{AfricData <-}\StringTok{ }\KeywordTok{filter}\NormalTok{(gapminder,continent==}\StringTok{"Africa"}\NormalTok{) }\CommentTok{# extracts the observations from Africa}
\KeywordTok{dim}\NormalTok{(AfricData) }\CommentTok{# displays the size of the data}
\end{Highlighting}
\end{Shaded}

\begin{verbatim}
## [1] 624   6
\end{verbatim}

\begin{Shaded}
\begin{Highlighting}[]
\KeywordTok{head}\NormalTok{(AfricData,}\DecValTok{20}\NormalTok{)  }\CommentTok{# displays the first 20 rows of the extracted data}
\end{Highlighting}
\end{Shaded}

\begin{verbatim}
## # A tibble: 20 x 6
##    country continent  year lifeExp      pop gdpPercap
##    <fct>   <fct>     <int>   <dbl>    <int>     <dbl>
##  1 Algeria Africa     1952    43.1  9279525     2449.
##  2 Algeria Africa     1957    45.7 10270856     3014.
##  3 Algeria Africa     1962    48.3 11000948     2551.
##  4 Algeria Africa     1967    51.4 12760499     3247.
##  5 Algeria Africa     1972    54.5 14760787     4183.
##  6 Algeria Africa     1977    58.0 17152804     4910.
##  7 Algeria Africa     1982    61.4 20033753     5745.
##  8 Algeria Africa     1987    65.8 23254956     5681.
##  9 Algeria Africa     1992    67.7 26298373     5023.
## 10 Algeria Africa     1997    69.2 29072015     4797.
## 11 Algeria Africa     2002    71.0 31287142     5288.
## 12 Algeria Africa     2007    72.3 33333216     6223.
## 13 Angola  Africa     1952    30.0  4232095     3521.
## 14 Angola  Africa     1957    32.0  4561361     3828.
## 15 Angola  Africa     1962    34    4826015     4269.
## 16 Angola  Africa     1967    36.0  5247469     5523.
## 17 Angola  Africa     1972    37.9  5894858     5473.
## 18 Angola  Africa     1977    39.5  6162675     3009.
## 19 Angola  Africa     1982    39.9  7016384     2757.
## 20 Angola  Africa     1987    39.9  7874230     2430.
\end{verbatim}

Now, we would extract the observation for NIgeria only.

\begin{Shaded}
\begin{Highlighting}[]
\NormalTok{NigData <-}\StringTok{ }\KeywordTok{filter}\NormalTok{(gapminder,country==}\StringTok{"Nigeria"}\NormalTok{)}
\KeywordTok{dim}\NormalTok{(NigData)}
\end{Highlighting}
\end{Shaded}

\begin{verbatim}
## [1] 12  6
\end{verbatim}

\begin{Shaded}
\begin{Highlighting}[]
\KeywordTok{head}\NormalTok{(NigData)}
\end{Highlighting}
\end{Shaded}

\begin{verbatim}
## # A tibble: 6 x 6
##   country continent  year lifeExp      pop gdpPercap
##   <fct>   <fct>     <int>   <dbl>    <int>     <dbl>
## 1 Nigeria Africa     1952    36.3 33119096     1077.
## 2 Nigeria Africa     1957    37.8 37173340     1101.
## 3 Nigeria Africa     1962    39.4 41871351     1151.
## 4 Nigeria Africa     1967    41.0 47287752     1015.
## 5 Nigeria Africa     1972    42.8 53740085     1698.
## 6 Nigeria Africa     1977    44.5 62209173     1982.
\end{verbatim}

Let us check how the expectancy of Nigerians have changed over the
years. We can plot this life expectancy over the years using
\textbf{geom\_line()} function.

\begin{Shaded}
\begin{Highlighting}[]
\NormalTok{NigData %>%}
\StringTok{  }\KeywordTok{ggplot}\NormalTok{( }\KeywordTok{aes}\NormalTok{(year,lifeExp)) +}\StringTok{ }
\StringTok{    }\KeywordTok{geom_line}\NormalTok{() +}\StringTok{ }
\StringTok{      }\KeywordTok{geom_point}\NormalTok{()}
\end{Highlighting}
\end{Shaded}

\includegraphics{Homework2_files/figure-latex/unnamed-chunk-51-1.pdf}

Now, let us plot Nigeria population over the years.

\begin{Shaded}
\begin{Highlighting}[]
\NormalTok{NigData %>%}\StringTok{   }\CommentTok{# load the extracted observation for Nigeria }
\StringTok{  }\KeywordTok{ggplot}\NormalTok{( }\KeywordTok{aes}\NormalTok{(year,pop)) +}\StringTok{ }\CommentTok{# calls the ggplot function}
\StringTok{    }\KeywordTok{geom_line}\NormalTok{() +}\StringTok{  }\CommentTok{# specifies the type of plot you want}
\StringTok{      }\KeywordTok{geom_area}\NormalTok{(}\DataTypeTok{fill=}\StringTok{'green'}\NormalTok{) }\CommentTok{# fill the area under the graph}
\end{Highlighting}
\end{Shaded}

\includegraphics{Homework2_files/figure-latex/unnamed-chunk-52-1.pdf}

Here, I have use the \textbf{geom\_area()} function to shade the area
under the curve.

We do the same for gdp per capital over the years.

\begin{Shaded}
\begin{Highlighting}[]
\NormalTok{NigData %>%}
\StringTok{  }\KeywordTok{ggplot}\NormalTok{( }\KeywordTok{aes}\NormalTok{(year,gdpPercap)) +}\StringTok{ }
\StringTok{    }\KeywordTok{geom_line}\NormalTok{() +}\StringTok{ }
\StringTok{      }\KeywordTok{geom_point}\NormalTok{()}
\end{Highlighting}
\end{Shaded}

\includegraphics{Homework2_files/figure-latex/unnamed-chunk-53-1.pdf}

Next, we check if there is relationship between the variables in the
observation for Nigeria.

Let us start by checking life expectancy and population over the years.

\begin{Shaded}
\begin{Highlighting}[]
\NormalTok{NigData %>%}
\StringTok{    }\KeywordTok{count}\NormalTok{(lifeExp,pop,year) %>%}
\StringTok{        }\KeywordTok{ggplot}\NormalTok{(}\KeywordTok{aes}\NormalTok{(}\DataTypeTok{x=}\NormalTok{lifeExp,}\DataTypeTok{y=}\NormalTok{pop)) +}\StringTok{ }
\StringTok{            }\KeywordTok{geom_point}\NormalTok{(}\KeywordTok{aes}\NormalTok{(}\DataTypeTok{color=}\NormalTok{year,}\DataTypeTok{size=}\NormalTok{year))+}
\StringTok{                }\KeywordTok{scale_y_log10}\NormalTok{()}
\end{Highlighting}
\end{Shaded}

\includegraphics{Homework2_files/figure-latex/unnamed-chunk-54-1.pdf}

From the graph, it looks like there is a positive relationship between
them. Let us check verify this checking the correlation between the two
variables

\begin{Shaded}
\begin{Highlighting}[]
\KeywordTok{cor}\NormalTok{(NigData$lifeExp,NigData$pop)  }\CommentTok{# computes correlation between lifeExp and pop}
\end{Highlighting}
\end{Shaded}

\begin{verbatim}
## [1] 0.853939
\end{verbatim}

This confirms that there is a stron positive relationship between them.
What of life expectancy and gdp per capital?

\begin{Shaded}
\begin{Highlighting}[]
\NormalTok{NigData %>%}
\StringTok{      }\KeywordTok{ggplot}\NormalTok{(}\KeywordTok{aes}\NormalTok{(}\DataTypeTok{x=}\NormalTok{lifeExp,}\DataTypeTok{y=}\NormalTok{gdpPercap)) +}\StringTok{ }
\StringTok{            }\KeywordTok{geom_point}\NormalTok{(}\KeywordTok{aes}\NormalTok{(}\DataTypeTok{color=}\NormalTok{year,}\DataTypeTok{size=}\NormalTok{year))+}
\StringTok{                }\KeywordTok{scale_y_log10}\NormalTok{()}
\end{Highlighting}
\end{Shaded}

\includegraphics{Homework2_files/figure-latex/unnamed-chunk-56-1.pdf}

It is hard to tell if there is a relationship from the plot. Let us
check my computing the correlation coefficient.

\begin{Shaded}
\begin{Highlighting}[]
\KeywordTok{cor}\NormalTok{(NigData$lifeExp,NigData$gdpPercap) }\CommentTok{# computes correlation between lifeExp and gdpPercap}
\end{Highlighting}
\end{Shaded}

\begin{verbatim}
## [1] 0.7360712
\end{verbatim}

There is a positive relationship between lifeExp and gdpPercap. Lastly,
we plot the population vs gdp per capital.

\begin{Shaded}
\begin{Highlighting}[]
\NormalTok{NigData %>%}
\StringTok{      }\KeywordTok{ggplot}\NormalTok{(}\KeywordTok{aes}\NormalTok{(}\DataTypeTok{x=}\NormalTok{pop,}\DataTypeTok{y=}\NormalTok{gdpPercap)) +}\StringTok{ }
\StringTok{            }\KeywordTok{geom_point}\NormalTok{(}\KeywordTok{aes}\NormalTok{(}\DataTypeTok{color=}\NormalTok{year,}\DataTypeTok{size=}\NormalTok{year))+}
\StringTok{                }\KeywordTok{scale_y_log10}\NormalTok{()}
\end{Highlighting}
\end{Shaded}

\includegraphics{Homework2_files/figure-latex/unnamed-chunk-58-1.pdf}

It is not easy to tell from the plot if a relationship exist, let us
verify with correlation coefficient.

\begin{Shaded}
\begin{Highlighting}[]
\KeywordTok{cor}\NormalTok{(NigData$pop,NigData$gdpPercap) }\CommentTok{# computes correlation between population and gdpPercap}
\end{Highlighting}
\end{Shaded}

\begin{verbatim}
## [1] 0.6928366
\end{verbatim}

\subsection{I want to do more}\label{i-want-to-do-more}

The analyst wanted to extract data for Rwanda and Afghanistan only and
they used the following code

\begin{Shaded}
\begin{Highlighting}[]
 \KeywordTok{filter}\NormalTok{(gapminder, country ==}\StringTok{ }\KeywordTok{c}\NormalTok{(}\StringTok{"Rwanda"}\NormalTok{,}\StringTok{"Afghanistan"}\NormalTok{))}
\end{Highlighting}
\end{Shaded}

\begin{verbatim}
## # A tibble: 12 x 6
##    country     continent  year lifeExp      pop gdpPercap
##    <fct>       <fct>     <int>   <dbl>    <int>     <dbl>
##  1 Afghanistan Asia       1957    30.3  9240934      821.
##  2 Afghanistan Asia       1967    34.0 11537966      836.
##  3 Afghanistan Asia       1977    38.4 14880372      786.
##  4 Afghanistan Asia       1987    40.8 13867957      852.
##  5 Afghanistan Asia       1997    41.8 22227415      635.
##  6 Afghanistan Asia       2007    43.8 31889923      975.
##  7 Rwanda      Africa     1952    40    2534927      493.
##  8 Rwanda      Africa     1962    43    3051242      597.
##  9 Rwanda      Africa     1972    44.6  3992121      591.
## 10 Rwanda      Africa     1982    46.2  5507565      882.
## 11 Rwanda      Africa     1992    23.6  7290203      737.
## 12 Rwanda      Africa     2002    43.4  7852401      786.
\end{verbatim}

\textbf{NO!} the analyst did not succeed. Because this only produced 12
rows, observations for only 6 years fro each of the countries instead of
12 for each, to give a total of 24 rows.

We see below that in the pagminder data, there are 12 rows for Rwanda
and also 12 for Afghanistan.

\begin{Shaded}
\begin{Highlighting}[]
 \KeywordTok{filter}\NormalTok{(gapminder, country ==}\StringTok{ "Rwanda"}\NormalTok{)}
\end{Highlighting}
\end{Shaded}

\begin{verbatim}
## # A tibble: 12 x 6
##    country continent  year lifeExp     pop gdpPercap
##    <fct>   <fct>     <int>   <dbl>   <int>     <dbl>
##  1 Rwanda  Africa     1952    40   2534927      493.
##  2 Rwanda  Africa     1957    41.5 2822082      540.
##  3 Rwanda  Africa     1962    43   3051242      597.
##  4 Rwanda  Africa     1967    44.1 3451079      511.
##  5 Rwanda  Africa     1972    44.6 3992121      591.
##  6 Rwanda  Africa     1977    45   4657072      670.
##  7 Rwanda  Africa     1982    46.2 5507565      882.
##  8 Rwanda  Africa     1987    44.0 6349365      848.
##  9 Rwanda  Africa     1992    23.6 7290203      737.
## 10 Rwanda  Africa     1997    36.1 7212583      590.
## 11 Rwanda  Africa     2002    43.4 7852401      786.
## 12 Rwanda  Africa     2007    46.2 8860588      863.
\end{verbatim}

\begin{Shaded}
\begin{Highlighting}[]
 \KeywordTok{filter}\NormalTok{(gapminder, country ==}\StringTok{ "Afghanistan"}\NormalTok{)}
\end{Highlighting}
\end{Shaded}

\begin{verbatim}
## # A tibble: 12 x 6
##    country     continent  year lifeExp      pop gdpPercap
##    <fct>       <fct>     <int>   <dbl>    <int>     <dbl>
##  1 Afghanistan Asia       1952    28.8  8425333      779.
##  2 Afghanistan Asia       1957    30.3  9240934      821.
##  3 Afghanistan Asia       1962    32.0 10267083      853.
##  4 Afghanistan Asia       1967    34.0 11537966      836.
##  5 Afghanistan Asia       1972    36.1 13079460      740.
##  6 Afghanistan Asia       1977    38.4 14880372      786.
##  7 Afghanistan Asia       1982    39.9 12881816      978.
##  8 Afghanistan Asia       1987    40.8 13867957      852.
##  9 Afghanistan Asia       1992    41.7 16317921      649.
## 10 Afghanistan Asia       1997    41.8 22227415      635.
## 11 Afghanistan Asia       2002    42.1 25268405      727.
## 12 Afghanistan Asia       2007    43.8 31889923      975.
\end{verbatim}

The follow code can be used to extract all the observations for Rwanda
and Afghanistan only from the gapminder data.

\begin{Shaded}
\begin{Highlighting}[]
 \KeywordTok{filter}\NormalTok{(gapminder, country ==}\StringTok{ "Rwanda"}  \NormalTok{|}\StringTok{ }\NormalTok{country ==}\StringTok{  "Afghanistan"} \NormalTok{)}
\end{Highlighting}
\end{Shaded}

\begin{verbatim}
## # A tibble: 24 x 6
##    country     continent  year lifeExp      pop gdpPercap
##    <fct>       <fct>     <int>   <dbl>    <int>     <dbl>
##  1 Afghanistan Asia       1952    28.8  8425333      779.
##  2 Afghanistan Asia       1957    30.3  9240934      821.
##  3 Afghanistan Asia       1962    32.0 10267083      853.
##  4 Afghanistan Asia       1967    34.0 11537966      836.
##  5 Afghanistan Asia       1972    36.1 13079460      740.
##  6 Afghanistan Asia       1977    38.4 14880372      786.
##  7 Afghanistan Asia       1982    39.9 12881816      978.
##  8 Afghanistan Asia       1987    40.8 13867957      852.
##  9 Afghanistan Asia       1992    41.7 16317921      649.
## 10 Afghanistan Asia       1997    41.8 22227415      635.
## # ... with 14 more rows
\end{verbatim}

\subsubsection{\texorpdfstring{\textbf{kableExtra}}{kableExtra}}\label{kableextra}

Let us install the \textbf{kableExtra} package

\begin{verbatim}
install.packages("kableExtra")
\end{verbatim}

next, we load the library \textbf{knitr} and \textbf{knitr}

\begin{Shaded}
\begin{Highlighting}[]
\KeywordTok{library}\NormalTok{(knitr)}
\KeywordTok{library}\NormalTok{(kableExtra)}
\end{Highlighting}
\end{Shaded}

Let us use the \textbf{kable()} function to display the gapminder data.
This function is slow when it is used on the gapminder data, I be using
another data to illustrate how it works.

Data of death rates in Virginia (1940).

\begin{Shaded}
\begin{Highlighting}[]
\KeywordTok{head}\NormalTok{(VADeaths)  }\CommentTok{# displays first few rows of data}
\end{Highlighting}
\end{Shaded}

\begin{verbatim}
##       Rural Male Rural Female Urban Male Urban Female
## 50-54       11.7          8.7       15.4          8.4
## 55-59       18.1         11.7       24.3         13.6
## 60-64       26.9         20.3       37.0         19.3
## 65-69       41.0         30.9       54.6         35.1
## 70-74       66.0         54.3       71.1         50.0
\end{verbatim}

\begin{Shaded}
\begin{Highlighting}[]
\KeywordTok{dim}\NormalTok{(VADeaths)  }\CommentTok{# displays the dimension}
\end{Highlighting}
\end{Shaded}

\begin{verbatim}
## [1] 5 4
\end{verbatim}

Note: The result for the kable function makes it difficult for my
\emph{github\_document} to run so I have displayed them in
\emph{html\_document} and commented the code here.

Let us use the \textbf{kable()} to display the

\begin{verbatim}
VADeaths %>%
  kable() %>%
   kable_styling() 
\end{verbatim}

It looks nice than what you will get using \textbf{head()} function

\begin{verbatim}
VADeaths %>%
  kable() %>%
   kable_styling(bootstrap_options = c("striped")) 
\end{verbatim}

We can also make the table smaller.

\begin{verbatim}
VADeaths %>%
  kable() %>%
   kable_styling(bootstrap_options = "striped", full_width = F) 
\end{verbatim}

Align the table to the left:

\begin{verbatim}
VADeaths %>%
  kable() %>%
   kable_styling(bootstrap_options = "striped", full_width = F, position="left") 
\end{verbatim}

\begin{verbatim}
VADeaths %>%
  kable() %>%
   kable_styling(bootstrap_options = "striped", full_width = F, position="right") 
\end{verbatim}

Adjusting the font size:

\begin{verbatim}
VADeaths %>%
  kable() %>%
   kable_styling(bootstrap_options = "striped", full_width = F, position="right", font_size=8) 
\end{verbatim}


\end{document}
